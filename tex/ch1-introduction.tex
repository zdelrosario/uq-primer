\documentclass[../primer.tex]{subfiles}

\begin{document}

\chapter{Introduction}
\lettrine[findent=2pt]{\textbf{W}}{illiam} Gibson's masterful short story
\emph{The Gernsback Continuum}\cite{gibson1986chrome} follows a man haunted by
an idea. His protagonist is pursued by \emph{semiotic ghosts}, lingering packets
of meaning, there in the form of ray-gun spacemen and ten-engine airships sprung
from the mind of legendary science fiction author Hugo Gernsback. This
character's dilemma is Gibson's reaction to the starry-eyed view of science
fiction in the 1970's -- a view stuck in the chrome-plated, streamlined future
of the 1930's which could never be.

We as engineers in the early 2000's are haunted by a different idea, that of an
engineering certainty which never existed. Our tools are predicated on a
deterministic, omniscent view of the universe which is woefully at odds with the
reality of our occupation. Our traditional approach is to make uncertainty
irrelevant -- to pile on margin and arbitrary safety factors until we creep away
from failure. This safety comes at the expense of inefficient, expensive
designs, and a lack of understanding of the relevant uncertainties, precluding
improvement.

This book is an attempt at exoricism. Rather than making it irrelevant, we aim
to \emph{Manage Uncertainty}.

\section{Purpose}
This book is foremost a \emph{primer}. We shall not attempt to cover an
exhaustive treatment of all relevant material, but rather shall seek to
understand important concepts, and to grasp some useful tools.

The topic is uncertainty, interpreted broadly. Humans, as a species, are
generally quite bad at managing uncertainty. At best it tends to make us
nervous: Think of the anxiety that strikes when moving to a new city, or asking
a stranger on a date. If the outcomes were certain -- if finding new friends
were guaranteed or our desired would accept -- there would be no anxiety. At
worst, we tend to ignore uncertainty: Think of politicians who make
self-assured\footnote{Often self-contradictory...} assertions.

Anxiety prevents us from taking action. Ignorance prevents us from learning. The
foremost objective of this primer is to catalyze a \emph{change in perspective};
to convince you, dear reader, that confronting uncertainty is useful, possible,
and necessary.

This book is intended for students of engineering, either early graduate or
advanced undergraduate. The content will make modest demands on your background;
we will assume fluency with differential and integral calculus. Regretfully,
many engineering curricula shortchange probability and (especially) statistics.
Both are powerful tools for handling uncertainty, and we will make thorough use
of these disciplines. Two appendices are provided to cover the bare essentials
of probability and statistics.

\section{How to use this book}

\section{Acknowledgements}


\end{document}
