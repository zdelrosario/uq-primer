\documentclass[../primer.tex]{subfiles}

\begin{document}

% --------------------------------------------------
\chapter{Introduction}
% --------------------------------------------------
\lettrine[findent=2pt]{\textbf{W}}{illiam} Gibson's masterful short story
\emph{The Gernsback Continuum}\cite{gibson1986chrome} follows a man haunted by
an idea. His protagonist is pursued by \emph{semiotic ghosts}, lingering packets
of meaning, there in the form of ray-gun spacemen and ten-engine airships sprung
from the mind of legendary science fiction author Hugo Gernsback. This
character's dilemma is Gibson's reaction to the starry-eyed view of science
fiction in the 1970's -- a view stuck in the chrome-plated, streamlined future
of the 1930's which could never be.

We as engineers in the early 2000's are haunted by a different idea, that of an
engineering certainty which never existed. Our tools are predicated on a
deterministic, omniscient view of the universe which is woefully at odds with the
reality of our occupation. Our traditional approach is to make uncertainty
irrelevant -- to pile on margin and arbitrary safety factors until we creep away
from failure. This safety comes at the expense of inefficient, expensive
designs, and a lack of understanding of the relevant uncertainties, precluding
improvement. Underlying this deterministic framework is a sentiment of
revulsion, perhaps even fear.

This book is an attempt at exorcism. Rather than making it irrelevant, or
becoming paralyzed by anxiety, we aim to \emph{Embrace Uncertainty}.

\section{Purpose}
% --------------------------------------------------
This book is foremost a \emph{primer}. We shall not attempt to cover an
exhaustive treatment of all relevant material, but rather shall seek to
understand important concepts, and to grasp some useful tools.

The topic is uncertainty, interpreted broadly. Humans, as a species, are
generally quite bad at managing uncertainty. At best it tends to make us
nervous: Think of the anxiety that strikes when moving to a new city, or asking
a stranger on a date. If the outcomes were certain -- if finding new friends
were guaranteed or our desired would accept -- there would be no anxiety. At
worst, we tend to ignore uncertainty: Think of politicians who make
self-assured\footnote{Often self-contradictory...} assertions.

Anxiety prevents us from taking action. Ignorance prevents us from learning. The
foremost objective of this primer is to catalyze a \emph{change in perspective};
to convince you, dear reader, that confronting uncertainty is useful, possible,
and necessary.

Furthermore, uncertainty is \emph{exciting}, and helps contribute to a rich,
well-lived life. Think of how boring a drama would be if all the plot-points
were spelled out before the first establishing shot -- amateur critics of films
often lament when a movie is predictable. \emph{Reframing} the examples from
before, socializing is exciting \emph{because} it is uncertain. As Richard
Powers is fond of saying, we should
\href{https://socialdance.stanford.edu/Syllabi/vertlateral.htm}{welcome chance
  encounters}. Of course, chance encounters in engineering design are quite
unlike those in social dance -- a misstep in engineering can cost lives.

With the above waxing philosophic complete, we can fully understand our mission
of \emph{embracing} uncertainty as akin to living in the
\href{https://qz.com/884448/every-successful-relationship-is-successful-for-the-same-exact-reasons/?utm_source=FBP011317_1}{ideal}\footnote{non-Hollywood}
relationship. Just as a successful relationship must be built on mutual respect
and continual growth, we as engineers must embrace the uncertainty endemic to
our chosen profession, respect and appreciate the unknowns that face us, and
continually grow in our understanding. Of course, we ought not take this
metaphor too far...

This book is intended for students of engineering, either early graduate or
advanced undergraduate. The content will make modest demands on your background;
we will assume fluency with differential and integral calculus. Regretfully,
many engineering curricula shortchange probability and (especially) statistics.
Both are powerful tools for handling uncertainty, and we will make thorough use
of these disciplines. Two appendices are provided to cover the bare essentials
of probability and statistics.

\section{How to use this book}
% --------------------------------------------------

\section{Acknowledgements}
% --------------------------------------------------
A great many people have influenced my thoughts on uncertainty, and their
influence is reflected in this book. Mike Baiocchi first introduced me to the
framing of \emph{embracing} uncertainty, within the context of the
reproducibility crisis. It is a testament to the power of language that a simple
little phrase can spur a fundamental change in one's philosophy, and I have Mike
to thank for this particular thought-germ; one that is still mutating my
understanding of the subject.

\end{document}
