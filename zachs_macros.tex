% Zach del Rosario's LaTeX macros (zdelrosario@outlook.com)
% Inspired by Paul Constantine, Art Owen
% Thanks to David Carlisle for writing the numdef package,
%    which makes the fraction definitions possible!

% Use % Zach del Rosario's LaTeX macros (zdelrosario@outlook.com)
% Inspired by Paul Constantine, Art Owen
% Thanks to David Carlisle for writing the numdef package,
%    which makes the fraction definitions possible!

% Use % Zach del Rosario's LaTeX macros (zdelrosario@outlook.com)
% Inspired by Paul Constantine, Art Owen
% Thanks to David Carlisle for writing the numdef package,
%    which makes the fraction definitions possible!

% Use % Zach del Rosario's LaTeX macros (zdelrosario@outlook.com)
% Inspired by Paul Constantine, Art Owen
% Thanks to David Carlisle for writing the numdef package,
%    which makes the fraction definitions possible!

% Use \input{zachs_macros} in preamble of a latex document

% --------------------------------------------------
% Use some package dependencies
% --------------------------------------------------
\usepackage{amsmath}  % for \boldsymbol, etc.
\usepackage{amsfonts} % for \mathbb, etc.
\usepackage{scalerel,stackengine} % for \reallywidehat{}
\usepackage{graphicx} % for \includegraphics
\usepackage{caption}  % for captioning
\usepackage{mathtools}% for \ceil and \floor
\usepackage{forest}   % for forest environment

% --------------------------------------------------
% Figures and tables
% --------------------------------------------------
% Could use as-is; better for pattern matching

% Image Macro: \img{filename}{caption}
\newcommand{\img}[2]{
	\begin{figure}[H]
	\centering
	\includegraphics[width=0.6\textwidth]{../images/#1}   % first argument is the file
	\caption{#2}                  % second argument is caption
	\label{fig:#1}                % generate label from first argument
	\end{figure} }

% Double Image Macro: \img{file1}{file2}{caption1}{caption2}
\newcommand{\imgtwo}[4]{
	\begin{figure}
	\centering
	\begin{minipage}{.5\textwidth}
		\centering
		\includegraphics[width=0.9\linewidth]{../images/#1}
		\captionof{figure}{#3}
		\label{fig:#1}
	\end{minipage}%
	\begin{minipage}{.5\textwidth}
		\centering
		\includegraphics[width=0.9\linewidth]{../images/#2}
		\captionof{figure}{#4}
		\label{fig:#2}
	\end{minipage}
	\end{figure}
}

% Table Macro: \tab{filename}{caption}
\newcommand{\tab}[2]{
	\begin{table}[H]
	\centering
	\input{#1} 	% first argument is filename
	\caption{#2} 			% second argument is caption
	\label{tab:#1} 			% generate label from filename
	\end{table}
}

% --------------------------------------------------
% Common sets
% --------------------------------------------------
% Lovingly ripped from Art Owen
\def\reals{\mathbb{R}} % Real number symbol
\def\integers{\mathbb{Z}} % Integer symbol
\def\rationals{\mathbb{Q}} % Rational numbers
\def\naturals{\mathbb{N}} % Natural numbers
\def\complex{\mathbb{C}} % Complex numbers
% With exponent
\def\R#1{\mathbb{R}^{#1}}
\def\Z#1{\mathbb{Z}^{#1}}
\def\Q#1{\mathbb{Q}^{#1}}
\def\N#1{\mathbb{N}^{#1}}
\def\C#1{\mathbb{C}^{#1}}

% --------------------------------------------------
% Vectors and Matrices
% --------------------------------------------------
% Vector symbol macros
\newcommand{\vsym}[1]{\boldsymbol{#1}}
\def\v#1{\vsym{#1}} % \v{x} for vector symbol
% Quick letter vectors
\newcommand{\va}{\boldsymbol{a}}
\newcommand{\vb}{\boldsymbol{b}}
\newcommand{\vc}{\boldsymbol{c}}
\newcommand{\vd}{\boldsymbol{d}}
\newcommand{\ve}{\boldsymbol{e}}
\newcommand{\vf}{\boldsymbol{f}}
\newcommand{\vg}{\boldsymbol{g}}
\newcommand{\vh}{\boldsymbol{h}}
\newcommand{\vi}{\boldsymbol{i}}
\newcommand{\vj}{\boldsymbol{j}}
\newcommand{\vk}{\boldsymbol{k}}
\newcommand{\vl}{\boldsymbol{l}}
\newcommand{\vm}{\boldsymbol{m}}
\newcommand{\vn}{\boldsymbol{n}}
\newcommand{\vo}{\boldsymbol{o}}
\newcommand{\vp}{\boldsymbol{p}}
\newcommand{\vq}{\boldsymbol{q}}
\newcommand{\vr}{\boldsymbol{r}}
\newcommand{\vs}{\boldsymbol{s}}
\newcommand{\vt}{\boldsymbol{t}}
\newcommand{\vu}{\boldsymbol{u}}
\newcommand{\vv}{\boldsymbol{v}}
\newcommand{\vw}{\boldsymbol{w}}
\newcommand{\vx}{\boldsymbol{x}}
\newcommand{\vy}{\boldsymbol{y}}
\newcommand{\vz}{\boldsymbol{z}}

% Tilde shortcut
\newcommand{\tl}[1]{\tilde{#1}}
% Vector symbol + tilde macros
\newcommand{\vta}{\tilde{\boldsymbol{a}}}
\newcommand{\vtb}{\tilde{\boldsymbol{b}}}
\newcommand{\vtc}{\tilde{\boldsymbol{c}}}
\newcommand{\vtd}{\tilde{\boldsymbol{d}}}
\newcommand{\vte}{\tilde{\boldsymbol{e}}}
\newcommand{\vtf}{\tilde{\boldsymbol{f}}}
\newcommand{\vtg}{\tilde{\boldsymbol{g}}}
\newcommand{\vth}{\tilde{\boldsymbol{h}}}
\newcommand{\vti}{\tilde{\boldsymbol{i}}}
\newcommand{\vtj}{\tilde{\boldsymbol{j}}}
\newcommand{\vtk}{\tilde{\boldsymbol{k}}}
\newcommand{\vtl}{\tilde{\boldsymbol{l}}}
\newcommand{\vtm}{\tilde{\boldsymbol{m}}}
\newcommand{\vtn}{\tilde{\boldsymbol{n}}}
\newcommand{\vto}{\tilde{\boldsymbol{o}}}
\newcommand{\vtp}{\tilde{\boldsymbol{p}}}
\newcommand{\vtq}{\tilde{\boldsymbol{q}}}
\newcommand{\vtr}{\tilde{\boldsymbol{r}}}
\newcommand{\vts}{\tilde{\boldsymbol{s}}}
\newcommand{\vtt}{\tilde{\boldsymbol{t}}}
\newcommand{\vtu}{\tilde{\boldsymbol{u}}}
\newcommand{\vtv}{\tilde{\boldsymbol{v}}}
\newcommand{\vtw}{\tilde{\boldsymbol{w}}}
\newcommand{\vtx}{\tilde{\boldsymbol{x}}}
\newcommand{\vty}{\tilde{\boldsymbol{y}}}
\newcommand{\vtz}{\tilde{\boldsymbol{z}}}

% Vector symbol + hat macros
\newcommand{\vha}{\hat{\boldsymbol{a}}}
\newcommand{\vhb}{\hat{\boldsymbol{b}}}
\newcommand{\vhc}{\hat{\boldsymbol{c}}}
\newcommand{\vhd}{\hat{\boldsymbol{d}}}
\newcommand{\vhe}{\hat{\boldsymbol{e}}}
\newcommand{\vhf}{\hat{\boldsymbol{f}}}
\newcommand{\vhg}{\hat{\boldsymbol{g}}}
\newcommand{\vhh}{\hat{\boldsymbol{h}}}
\newcommand{\vhi}{\hat{\boldsymbol{i}}}
\newcommand{\vhj}{\hat{\boldsymbol{j}}}
\newcommand{\vhk}{\hat{\boldsymbol{k}}}
\newcommand{\vhl}{\hat{\boldsymbol{l}}}
\newcommand{\vhm}{\hat{\boldsymbol{m}}}
\newcommand{\vhn}{\hat{\boldsymbol{n}}}
\newcommand{\vho}{\hat{\boldsymbol{o}}}
\newcommand{\vhp}{\hat{\boldsymbol{p}}}
\newcommand{\vhq}{\hat{\boldsymbol{q}}}
\newcommand{\vhr}{\hat{\boldsymbol{r}}}
\newcommand{\vhs}{\hat{\boldsymbol{s}}}
\newcommand{\vht}{\hat{\boldsymbol{t}}}
\newcommand{\vhu}{\hat{\boldsymbol{u}}}
\newcommand{\vhv}{\hat{\boldsymbol{v}}}
\newcommand{\vhw}{\hat{\boldsymbol{w}}}
\newcommand{\vhx}{\hat{\boldsymbol{x}}}
\newcommand{\vhy}{\hat{\boldsymbol{y}}}
\newcommand{\vhz}{\hat{\boldsymbol{z}}}

% Matrix symbol
\newcommand{\msym}[1]{\boldsymbol{#1}}
\def\m#1{\msym{#1}} % short-shortcut

\newcommand{\mA}{\boldsymbol{A}}
\newcommand{\mB}{\boldsymbol{B}}
\newcommand{\mC}{\boldsymbol{C}}
\newcommand{\mD}{\boldsymbol{D}}
\newcommand{\mE}{\boldsymbol{E}}
\newcommand{\mF}{\boldsymbol{F}}
\newcommand{\mG}{\boldsymbol{G}}
\newcommand{\mH}{\boldsymbol{H}}
\newcommand{\mI}{\boldsymbol{I}}
\newcommand{\mJ}{\boldsymbol{J}}
\newcommand{\mK}{\boldsymbol{K}}
\newcommand{\mL}{\boldsymbol{L}}
\newcommand{\mM}{\boldsymbol{M}}
\newcommand{\mN}{\boldsymbol{N}}
\newcommand{\mO}{\boldsymbol{O}}
\newcommand{\mP}{\boldsymbol{P}}
\newcommand{\mQ}{\boldsymbol{Q}}
\newcommand{\mR}{\boldsymbol{R}}
\newcommand{\mS}{\boldsymbol{S}}
\newcommand{\mT}{\boldsymbol{T}}
\newcommand{\mU}{\boldsymbol{U}}
\newcommand{\mV}{\boldsymbol{V}}
\newcommand{\mW}{\boldsymbol{W}}
\newcommand{\mX}{\boldsymbol{X}}
\newcommand{\mY}{\boldsymbol{Y}}
\newcommand{\mZ}{\boldsymbol{Z}}

% Tilde over letter
\newcommand{\tla}{\tilde{a}}
\newcommand{\tlb}{\tilde{b}}
\newcommand{\tlc}{\tilde{c}}
\newcommand{\tld}{\tilde{d}}
\newcommand{\tle}{\tilde{e}}
\newcommand{\tlf}{\tilde{f}}
\newcommand{\tlg}{\tilde{g}}
\newcommand{\tlh}{\tilde{h}}
\newcommand{\tli}{\tilde{i}}
\newcommand{\tlj}{\tilde{j}}
\newcommand{\tlk}{\tilde{k}}
\newcommand{\tll}{\tilde{l}}
\newcommand{\tlm}{\tilde{m}}
\newcommand{\tln}{\tilde{n}}
\newcommand{\tlo}{\tilde{o}}
\newcommand{\tlp}{\tilde{p}}
\newcommand{\tlq}{\tilde{q}}
\newcommand{\tlr}{\tilde{r}}
\newcommand{\tls}{\tilde{s}}
\newcommand{\tlt}{\tilde{t}}
\newcommand{\tlu}{\tilde{u}}
\newcommand{\tlv}{\tilde{v}}
\newcommand{\tlw}{\tilde{w}}
\newcommand{\tlx}{\tilde{x}}
\newcommand{\tly}{\tilde{y}}
\newcommand{\tlz}{\tilde{z}}

% Caligraphic symbol
\def\c#1{\mathcal{#1}} % short-shortcut

\newcommand{\cA}{\mathcal{A}}
\newcommand{\cB}{\mathcal{B}}
\newcommand{\cC}{\mathcal{C}}
\newcommand{\cD}{\mathcal{D}}
\newcommand{\cE}{\mathcal{E}}
\newcommand{\cF}{\mathcal{F}}
\newcommand{\cG}{\mathcal{G}}
\newcommand{\cH}{\mathcal{H}}
\newcommand{\cI}{\mathcal{I}}
\newcommand{\cJ}{\mathcal{J}}
\newcommand{\cK}{\mathcal{K}}
\newcommand{\cL}{\mathcal{L}}
\newcommand{\cM}{\mathcal{M}}
\newcommand{\cN}{\mathcal{N}}
\newcommand{\cO}{\mathcal{O}}
\newcommand{\cP}{\mathcal{P}}
\newcommand{\cQ}{\mathcal{Q}}
\newcommand{\cR}{\mathcal{R}}
\newcommand{\cS}{\mathcal{S}}
\newcommand{\cT}{\mathcal{T}}
\newcommand{\cU}{\mathcal{U}}
\newcommand{\cV}{\mathcal{V}}
\newcommand{\cW}{\mathcal{W}}
\newcommand{\cX}{\mathcal{X}}
\newcommand{\cY}{\mathcal{Y}}
\newcommand{\cZ}{\mathcal{Z}}
% --------------------------------------------------
% Common probability symbols
% --------------------------------------------------
% Lovingly ripped from Art Owen
\newcommand{\mrm}{\mathrm}
\def\E{\mathbb{E}} % Expectation symbol
\def\Earg#1{\E\left[{#1}\right]}
\def\Esubarg#1#2{\E_{#1}\left[{#2}\right]}
\def\P{\mathbb{P}} % Probability symbol
\def\Parg#1{\P\left({#1}\right)}
\def\Psubarg#1#2{\P_{#1}\left[{#2}\right]}
\def\Cov{\mrm{Cov}} % Covariance symbol
\def\Covarg#1{\Cov\left[{#1}\right]}
\def\Covsubarg#1#2{\Cov_{#1}\left[{#2}\right]}
\newcommand{\family}{\mathcal{P}} % probability family / statistical model
\newcommand{\iid}{\stackrel{\mathrm{iid}}{\sim}}
\newcommand{\ind}{\stackrel{\mathrm{ind}}{\sim}}
\def\V{\mathrm{V}} % Variance symbol

% --------------------------------------------------
% Misc
% --------------------------------------------------
% Indicator function
\def\i1{\mathbb{1}}

% Angle bracket average
\newcommand{\avg}[1]{\left\langle#1\right\rangle}

% Markless footnote
% https://tex.stackexchange.com/questions/30720/footnote-without-a-marker
\newcommand\blfootnote[1]{%
  \begingroup
  \renewcommand\thefootnote{}\footnote{#1}%
  \addtocounter{footnote}{-1}%
  \endgroup
}

% Floor and ceiling
% http://tex.stackexchange.com/questions/118173/how-to-write-ceil-and-floor-in-latex
\DeclarePairedDelimiter\ceil{\lceil}{\rceil}
\DeclarePairedDelimiter\floor{\lfloor}{\rfloor}

% Comical & useful reallywidehat
\stackMath
\newcommand\reallywidehat[1]{%
\savestack{\tmpbox}{\stretchto{%
  \scaleto{%
    \scalerel*[\widthof{\ensuremath{#1}}]{\kern-.6pt\bigwedge\kern-.6pt}%
    {\rule[-\textheight/2]{1ex}{\textheight}}%WIDTH-LIMITED BIG WEDGE
  }{\textheight}%
}{0.5ex}}%
\stackon[1pt]{#1}{\tmpbox}%
}

% Comical & useful reallywideparen
\newcommand\reallywideparen[1]{%
\begin{array}{c}
\stretchto{
  \scaleto{
    \scalerel*[\widthof{#1}]{\frown}
    {\rule[-\textheight/2]{1ex}{\textheight}} %WIDTH-LIMITED BIG WEDGE
  }{1.25\textheight} % THIS STRETCHES THE WEDGE A LITTLE EXTRA WIDE
}{0.5ex}\\           % THIS SQUEEZES THE WEDGE TO 0.5ex HEIGHT
#1\\                   % THIS STACKS THE WEDGE ATOP THE ARGUMENT
\rule{0ex}{.01ex}
\end{array}
}
% Useful for debugging; prints to document whether command has been defined already
% Via: http://tex.stackexchange.com/questions/30483/how-can-i-check-in-latex-or-plain-tex-whether-a-command-exists-by-name
\newcommand{\checkfor}[1]{%
  \ifcsname#1\endcsname%
    ... command '#1' exists ...%
  \else%
    ... command '#1' does not exist ...%
  \fi%
}

% Use a forest environment to depict a directory tree
% https://tex.stackexchange.com/questions/5073/making-a-simple-directory-tree
\newcommand{\ftree}[1]{%
\begin{forest}
for tree={
    font=\ttfamily,
    grow'=0,
    child anchor=west,
    parent anchor=south,
    anchor=west,
    calign=first,
    edge path={
      \noexpand\path [draw, \forestoption{edge}]
      (!u.south west) +(7.5pt,0) |- node[fill,inner sep=1.25pt] {} (.child anchor)\forestoption{edge label};
    },
    before typesetting nodes={
      if n=1
        {insert before={[,phantom]}}
        {}
    },
    fit=band,
    before computing xy={l=15pt},
  }%
  #1%
\end{forest}
}

% --------------------------------------------------
% Fractions
% --------------------------------------------------
\usepackage{numdef}   % prefixing with \num allows numbers at end of \newcommand name
\num\newcommand{\f12}{\frac{1}{2}}
\num\newcommand{\f13}{\frac{1}{3}}
\num\newcommand{\f14}{\frac{1}{4}}
\num\newcommand{\f15}{\frac{1}{5}}
\num\newcommand{\f16}{\frac{1}{6}}
\num\newcommand{\f17}{\frac{1}{7}}
\num\newcommand{\f18}{\frac{1}{8}}
\num\newcommand{\f19}{\frac{1}{9}}

\num\newcommand{\f22}{\frac{2}{2}}
\num\newcommand{\f23}{\frac{2}{3}}
\num\newcommand{\f24}{\frac{2}{4}}
\num\newcommand{\f25}{\frac{2}{5}}
\num\newcommand{\f26}{\frac{2}{6}}
\num\newcommand{\f27}{\frac{2}{7}}
\num\newcommand{\f28}{\frac{2}{8}}
\num\newcommand{\f29}{\frac{2}{9}}

\num\newcommand{\f32}{\frac{3}{2}}
\num\newcommand{\f33}{\frac{3}{3}}
\num\newcommand{\f34}{\frac{3}{4}}
\num\newcommand{\f35}{\frac{3}{5}}
\num\newcommand{\f36}{\frac{3}{6}}
\num\newcommand{\f37}{\frac{3}{7}}
\num\newcommand{\f38}{\frac{3}{8}}
\num\newcommand{\f39}{\frac{3}{9}}

\num\newcommand{\f42}{\frac{4}{2}}
\num\newcommand{\f43}{\frac{4}{3}}
\num\newcommand{\f44}{\frac{4}{4}}
\num\newcommand{\f45}{\frac{4}{5}}
\num\newcommand{\f46}{\frac{4}{6}}
\num\newcommand{\f47}{\frac{4}{7}}
\num\newcommand{\f48}{\frac{4}{8}}
\num\newcommand{\f49}{\frac{4}{9}}

\num\newcommand{\f52}{\frac{5}{2}}
\num\newcommand{\f53}{\frac{5}{3}}
\num\newcommand{\f54}{\frac{5}{4}}
\num\newcommand{\f55}{\frac{5}{5}}
\num\newcommand{\f56}{\frac{5}{6}}
\num\newcommand{\f57}{\frac{5}{7}}
\num\newcommand{\f58}{\frac{5}{8}}
\num\newcommand{\f59}{\frac{5}{9}}

\num\newcommand{\f62}{\frac{6}{2}}
\num\newcommand{\f63}{\frac{6}{3}}
\num\newcommand{\f64}{\frac{6}{4}}
\num\newcommand{\f65}{\frac{6}{5}}
\num\newcommand{\f66}{\frac{6}{6}}
\num\newcommand{\f67}{\frac{6}{7}}
\num\newcommand{\f68}{\frac{6}{8}}
\num\newcommand{\f69}{\frac{6}{9}}

\num\newcommand{\f72}{\frac{7}{2}}
\num\newcommand{\f73}{\frac{7}{3}}
\num\newcommand{\f74}{\frac{7}{4}}
\num\newcommand{\f75}{\frac{7}{5}}
\num\newcommand{\f76}{\frac{7}{6}}
\num\newcommand{\f77}{\frac{7}{7}}
\num\newcommand{\f78}{\frac{7}{8}}
\num\newcommand{\f79}{\frac{7}{9}}

\num\newcommand{\f82}{\frac{8}{2}}
\num\newcommand{\f83}{\frac{8}{3}}
\num\newcommand{\f84}{\frac{8}{4}}
\num\newcommand{\f85}{\frac{8}{5}}
\num\newcommand{\f86}{\frac{8}{6}}
\num\newcommand{\f87}{\frac{8}{7}}
\num\newcommand{\f88}{\frac{8}{8}}
\num\newcommand{\f89}{\frac{8}{9}}

\num\newcommand{\f92}{\frac{9}{2}}
\num\newcommand{\f93}{\frac{9}{3}}
\num\newcommand{\f94}{\frac{9}{4}}
\num\newcommand{\f95}{\frac{9}{5}}
\num\newcommand{\f96}{\frac{9}{6}}
\num\newcommand{\f97}{\frac{9}{7}}
\num\newcommand{\f98}{\frac{9}{8}}
\num\newcommand{\f99}{\frac{9}{9}}

% --------------------------------------------------
% Cylindrical and Spherical operators
% --------------------------------------------------
% Cylindrical coordinates {r,\phi,z}
\num\newcommand{\cdiv1}[1]{\frac{1}{r}\frac{\partial}{\partial r}(r#1)}
\num\newcommand{\cdiv2}[1]{\frac{1}{r}\frac{\partial}{\partial\phi}(#1)}
\num\newcommand{\cdiv3}[1]{\frac{\partial}{\partial z}(#1)}

%--------------------------------------------------
% matminux
%--------------------------------------------------
% https://tex.stackexchange.com/questions/75545/negative-sign-and-matrix-alignment
\newcommand*{\mm}{%
  \leavevmode
  \hphantom{0}%
  \llap{%
    \settowidth{\dimen0 }{$0$}%
    \resizebox{1.1\dimen0 }{\height}{$-$}%
  }%
}
 in preamble of a latex document

% --------------------------------------------------
% Use some package dependencies
% --------------------------------------------------
\usepackage{amsmath}  % for \boldsymbol, etc.
\usepackage{amsfonts} % for \mathbb, etc.
\usepackage{scalerel,stackengine} % for \reallywidehat{}
\usepackage{graphicx} % for \includegraphics
\usepackage{caption}  % for captioning
\usepackage{mathtools}% for \ceil and \floor
\usepackage{forest}   % for forest environment

% --------------------------------------------------
% Figures and tables
% --------------------------------------------------
% Could use as-is; better for pattern matching

% Image Macro: \img{filename}{caption}
\newcommand{\img}[2]{
	\begin{figure}[H]
	\centering
	\includegraphics[width=0.6\textwidth]{../images/#1}   % first argument is the file
	\caption{#2}                  % second argument is caption
	\label{fig:#1}                % generate label from first argument
	\end{figure} }

% Double Image Macro: \img{file1}{file2}{caption1}{caption2}
\newcommand{\imgtwo}[4]{
	\begin{figure}
	\centering
	\begin{minipage}{.5\textwidth}
		\centering
		\includegraphics[width=0.9\linewidth]{../images/#1}
		\captionof{figure}{#3}
		\label{fig:#1}
	\end{minipage}%
	\begin{minipage}{.5\textwidth}
		\centering
		\includegraphics[width=0.9\linewidth]{../images/#2}
		\captionof{figure}{#4}
		\label{fig:#2}
	\end{minipage}
	\end{figure}
}

% Table Macro: \tab{filename}{caption}
\newcommand{\tab}[2]{
	\begin{table}[H]
	\centering
	\input{#1} 	% first argument is filename
	\caption{#2} 			% second argument is caption
	\label{tab:#1} 			% generate label from filename
	\end{table}
}

% --------------------------------------------------
% Common sets
% --------------------------------------------------
% Lovingly ripped from Art Owen
\def\reals{\mathbb{R}} % Real number symbol
\def\integers{\mathbb{Z}} % Integer symbol
\def\rationals{\mathbb{Q}} % Rational numbers
\def\naturals{\mathbb{N}} % Natural numbers
\def\complex{\mathbb{C}} % Complex numbers
% With exponent
\def\R#1{\mathbb{R}^{#1}}
\def\Z#1{\mathbb{Z}^{#1}}
\def\Q#1{\mathbb{Q}^{#1}}
\def\N#1{\mathbb{N}^{#1}}
\def\C#1{\mathbb{C}^{#1}}

% --------------------------------------------------
% Vectors and Matrices
% --------------------------------------------------
% Vector symbol macros
\newcommand{\vsym}[1]{\boldsymbol{#1}}
\def\v#1{\vsym{#1}} % \v{x} for vector symbol
% Quick letter vectors
\newcommand{\va}{\boldsymbol{a}}
\newcommand{\vb}{\boldsymbol{b}}
\newcommand{\vc}{\boldsymbol{c}}
\newcommand{\vd}{\boldsymbol{d}}
\newcommand{\ve}{\boldsymbol{e}}
\newcommand{\vf}{\boldsymbol{f}}
\newcommand{\vg}{\boldsymbol{g}}
\newcommand{\vh}{\boldsymbol{h}}
\newcommand{\vi}{\boldsymbol{i}}
\newcommand{\vj}{\boldsymbol{j}}
\newcommand{\vk}{\boldsymbol{k}}
\newcommand{\vl}{\boldsymbol{l}}
\newcommand{\vm}{\boldsymbol{m}}
\newcommand{\vn}{\boldsymbol{n}}
\newcommand{\vo}{\boldsymbol{o}}
\newcommand{\vp}{\boldsymbol{p}}
\newcommand{\vq}{\boldsymbol{q}}
\newcommand{\vr}{\boldsymbol{r}}
\newcommand{\vs}{\boldsymbol{s}}
\newcommand{\vt}{\boldsymbol{t}}
\newcommand{\vu}{\boldsymbol{u}}
\newcommand{\vv}{\boldsymbol{v}}
\newcommand{\vw}{\boldsymbol{w}}
\newcommand{\vx}{\boldsymbol{x}}
\newcommand{\vy}{\boldsymbol{y}}
\newcommand{\vz}{\boldsymbol{z}}

% Tilde shortcut
\newcommand{\tl}[1]{\tilde{#1}}
% Vector symbol + tilde macros
\newcommand{\vta}{\tilde{\boldsymbol{a}}}
\newcommand{\vtb}{\tilde{\boldsymbol{b}}}
\newcommand{\vtc}{\tilde{\boldsymbol{c}}}
\newcommand{\vtd}{\tilde{\boldsymbol{d}}}
\newcommand{\vte}{\tilde{\boldsymbol{e}}}
\newcommand{\vtf}{\tilde{\boldsymbol{f}}}
\newcommand{\vtg}{\tilde{\boldsymbol{g}}}
\newcommand{\vth}{\tilde{\boldsymbol{h}}}
\newcommand{\vti}{\tilde{\boldsymbol{i}}}
\newcommand{\vtj}{\tilde{\boldsymbol{j}}}
\newcommand{\vtk}{\tilde{\boldsymbol{k}}}
\newcommand{\vtl}{\tilde{\boldsymbol{l}}}
\newcommand{\vtm}{\tilde{\boldsymbol{m}}}
\newcommand{\vtn}{\tilde{\boldsymbol{n}}}
\newcommand{\vto}{\tilde{\boldsymbol{o}}}
\newcommand{\vtp}{\tilde{\boldsymbol{p}}}
\newcommand{\vtq}{\tilde{\boldsymbol{q}}}
\newcommand{\vtr}{\tilde{\boldsymbol{r}}}
\newcommand{\vts}{\tilde{\boldsymbol{s}}}
\newcommand{\vtt}{\tilde{\boldsymbol{t}}}
\newcommand{\vtu}{\tilde{\boldsymbol{u}}}
\newcommand{\vtv}{\tilde{\boldsymbol{v}}}
\newcommand{\vtw}{\tilde{\boldsymbol{w}}}
\newcommand{\vtx}{\tilde{\boldsymbol{x}}}
\newcommand{\vty}{\tilde{\boldsymbol{y}}}
\newcommand{\vtz}{\tilde{\boldsymbol{z}}}

% Vector symbol + hat macros
\newcommand{\vha}{\hat{\boldsymbol{a}}}
\newcommand{\vhb}{\hat{\boldsymbol{b}}}
\newcommand{\vhc}{\hat{\boldsymbol{c}}}
\newcommand{\vhd}{\hat{\boldsymbol{d}}}
\newcommand{\vhe}{\hat{\boldsymbol{e}}}
\newcommand{\vhf}{\hat{\boldsymbol{f}}}
\newcommand{\vhg}{\hat{\boldsymbol{g}}}
\newcommand{\vhh}{\hat{\boldsymbol{h}}}
\newcommand{\vhi}{\hat{\boldsymbol{i}}}
\newcommand{\vhj}{\hat{\boldsymbol{j}}}
\newcommand{\vhk}{\hat{\boldsymbol{k}}}
\newcommand{\vhl}{\hat{\boldsymbol{l}}}
\newcommand{\vhm}{\hat{\boldsymbol{m}}}
\newcommand{\vhn}{\hat{\boldsymbol{n}}}
\newcommand{\vho}{\hat{\boldsymbol{o}}}
\newcommand{\vhp}{\hat{\boldsymbol{p}}}
\newcommand{\vhq}{\hat{\boldsymbol{q}}}
\newcommand{\vhr}{\hat{\boldsymbol{r}}}
\newcommand{\vhs}{\hat{\boldsymbol{s}}}
\newcommand{\vht}{\hat{\boldsymbol{t}}}
\newcommand{\vhu}{\hat{\boldsymbol{u}}}
\newcommand{\vhv}{\hat{\boldsymbol{v}}}
\newcommand{\vhw}{\hat{\boldsymbol{w}}}
\newcommand{\vhx}{\hat{\boldsymbol{x}}}
\newcommand{\vhy}{\hat{\boldsymbol{y}}}
\newcommand{\vhz}{\hat{\boldsymbol{z}}}

% Matrix symbol
\newcommand{\msym}[1]{\boldsymbol{#1}}
\def\m#1{\msym{#1}} % short-shortcut

\newcommand{\mA}{\boldsymbol{A}}
\newcommand{\mB}{\boldsymbol{B}}
\newcommand{\mC}{\boldsymbol{C}}
\newcommand{\mD}{\boldsymbol{D}}
\newcommand{\mE}{\boldsymbol{E}}
\newcommand{\mF}{\boldsymbol{F}}
\newcommand{\mG}{\boldsymbol{G}}
\newcommand{\mH}{\boldsymbol{H}}
\newcommand{\mI}{\boldsymbol{I}}
\newcommand{\mJ}{\boldsymbol{J}}
\newcommand{\mK}{\boldsymbol{K}}
\newcommand{\mL}{\boldsymbol{L}}
\newcommand{\mM}{\boldsymbol{M}}
\newcommand{\mN}{\boldsymbol{N}}
\newcommand{\mO}{\boldsymbol{O}}
\newcommand{\mP}{\boldsymbol{P}}
\newcommand{\mQ}{\boldsymbol{Q}}
\newcommand{\mR}{\boldsymbol{R}}
\newcommand{\mS}{\boldsymbol{S}}
\newcommand{\mT}{\boldsymbol{T}}
\newcommand{\mU}{\boldsymbol{U}}
\newcommand{\mV}{\boldsymbol{V}}
\newcommand{\mW}{\boldsymbol{W}}
\newcommand{\mX}{\boldsymbol{X}}
\newcommand{\mY}{\boldsymbol{Y}}
\newcommand{\mZ}{\boldsymbol{Z}}

% Tilde over letter
\newcommand{\tla}{\tilde{a}}
\newcommand{\tlb}{\tilde{b}}
\newcommand{\tlc}{\tilde{c}}
\newcommand{\tld}{\tilde{d}}
\newcommand{\tle}{\tilde{e}}
\newcommand{\tlf}{\tilde{f}}
\newcommand{\tlg}{\tilde{g}}
\newcommand{\tlh}{\tilde{h}}
\newcommand{\tli}{\tilde{i}}
\newcommand{\tlj}{\tilde{j}}
\newcommand{\tlk}{\tilde{k}}
\newcommand{\tll}{\tilde{l}}
\newcommand{\tlm}{\tilde{m}}
\newcommand{\tln}{\tilde{n}}
\newcommand{\tlo}{\tilde{o}}
\newcommand{\tlp}{\tilde{p}}
\newcommand{\tlq}{\tilde{q}}
\newcommand{\tlr}{\tilde{r}}
\newcommand{\tls}{\tilde{s}}
\newcommand{\tlt}{\tilde{t}}
\newcommand{\tlu}{\tilde{u}}
\newcommand{\tlv}{\tilde{v}}
\newcommand{\tlw}{\tilde{w}}
\newcommand{\tlx}{\tilde{x}}
\newcommand{\tly}{\tilde{y}}
\newcommand{\tlz}{\tilde{z}}

% Caligraphic symbol
\def\c#1{\mathcal{#1}} % short-shortcut

\newcommand{\cA}{\mathcal{A}}
\newcommand{\cB}{\mathcal{B}}
\newcommand{\cC}{\mathcal{C}}
\newcommand{\cD}{\mathcal{D}}
\newcommand{\cE}{\mathcal{E}}
\newcommand{\cF}{\mathcal{F}}
\newcommand{\cG}{\mathcal{G}}
\newcommand{\cH}{\mathcal{H}}
\newcommand{\cI}{\mathcal{I}}
\newcommand{\cJ}{\mathcal{J}}
\newcommand{\cK}{\mathcal{K}}
\newcommand{\cL}{\mathcal{L}}
\newcommand{\cM}{\mathcal{M}}
\newcommand{\cN}{\mathcal{N}}
\newcommand{\cO}{\mathcal{O}}
\newcommand{\cP}{\mathcal{P}}
\newcommand{\cQ}{\mathcal{Q}}
\newcommand{\cR}{\mathcal{R}}
\newcommand{\cS}{\mathcal{S}}
\newcommand{\cT}{\mathcal{T}}
\newcommand{\cU}{\mathcal{U}}
\newcommand{\cV}{\mathcal{V}}
\newcommand{\cW}{\mathcal{W}}
\newcommand{\cX}{\mathcal{X}}
\newcommand{\cY}{\mathcal{Y}}
\newcommand{\cZ}{\mathcal{Z}}
% --------------------------------------------------
% Common probability symbols
% --------------------------------------------------
% Lovingly ripped from Art Owen
\newcommand{\mrm}{\mathrm}
\def\E{\mathbb{E}} % Expectation symbol
\def\Earg#1{\E\left[{#1}\right]}
\def\Esubarg#1#2{\E_{#1}\left[{#2}\right]}
\def\P{\mathbb{P}} % Probability symbol
\def\Parg#1{\P\left({#1}\right)}
\def\Psubarg#1#2{\P_{#1}\left[{#2}\right]}
\def\Cov{\mrm{Cov}} % Covariance symbol
\def\Covarg#1{\Cov\left[{#1}\right]}
\def\Covsubarg#1#2{\Cov_{#1}\left[{#2}\right]}
\newcommand{\family}{\mathcal{P}} % probability family / statistical model
\newcommand{\iid}{\stackrel{\mathrm{iid}}{\sim}}
\newcommand{\ind}{\stackrel{\mathrm{ind}}{\sim}}
\def\V{\mathrm{V}} % Variance symbol

% --------------------------------------------------
% Misc
% --------------------------------------------------
% Indicator function
\def\i1{\mathbb{1}}

% Angle bracket average
\newcommand{\avg}[1]{\left\langle#1\right\rangle}

% Markless footnote
% https://tex.stackexchange.com/questions/30720/footnote-without-a-marker
\newcommand\blfootnote[1]{%
  \begingroup
  \renewcommand\thefootnote{}\footnote{#1}%
  \addtocounter{footnote}{-1}%
  \endgroup
}

% Floor and ceiling
% http://tex.stackexchange.com/questions/118173/how-to-write-ceil-and-floor-in-latex
\DeclarePairedDelimiter\ceil{\lceil}{\rceil}
\DeclarePairedDelimiter\floor{\lfloor}{\rfloor}

% Comical & useful reallywidehat
\stackMath
\newcommand\reallywidehat[1]{%
\savestack{\tmpbox}{\stretchto{%
  \scaleto{%
    \scalerel*[\widthof{\ensuremath{#1}}]{\kern-.6pt\bigwedge\kern-.6pt}%
    {\rule[-\textheight/2]{1ex}{\textheight}}%WIDTH-LIMITED BIG WEDGE
  }{\textheight}%
}{0.5ex}}%
\stackon[1pt]{#1}{\tmpbox}%
}

% Comical & useful reallywideparen
\newcommand\reallywideparen[1]{%
\begin{array}{c}
\stretchto{
  \scaleto{
    \scalerel*[\widthof{#1}]{\frown}
    {\rule[-\textheight/2]{1ex}{\textheight}} %WIDTH-LIMITED BIG WEDGE
  }{1.25\textheight} % THIS STRETCHES THE WEDGE A LITTLE EXTRA WIDE
}{0.5ex}\\           % THIS SQUEEZES THE WEDGE TO 0.5ex HEIGHT
#1\\                   % THIS STACKS THE WEDGE ATOP THE ARGUMENT
\rule{0ex}{.01ex}
\end{array}
}
% Useful for debugging; prints to document whether command has been defined already
% Via: http://tex.stackexchange.com/questions/30483/how-can-i-check-in-latex-or-plain-tex-whether-a-command-exists-by-name
\newcommand{\checkfor}[1]{%
  \ifcsname#1\endcsname%
    ... command '#1' exists ...%
  \else%
    ... command '#1' does not exist ...%
  \fi%
}

% Use a forest environment to depict a directory tree
% https://tex.stackexchange.com/questions/5073/making-a-simple-directory-tree
\newcommand{\ftree}[1]{%
\begin{forest}
for tree={
    font=\ttfamily,
    grow'=0,
    child anchor=west,
    parent anchor=south,
    anchor=west,
    calign=first,
    edge path={
      \noexpand\path [draw, \forestoption{edge}]
      (!u.south west) +(7.5pt,0) |- node[fill,inner sep=1.25pt] {} (.child anchor)\forestoption{edge label};
    },
    before typesetting nodes={
      if n=1
        {insert before={[,phantom]}}
        {}
    },
    fit=band,
    before computing xy={l=15pt},
  }%
  #1%
\end{forest}
}

% --------------------------------------------------
% Fractions
% --------------------------------------------------
\usepackage{numdef}   % prefixing with \num allows numbers at end of \newcommand name
\num\newcommand{\f12}{\frac{1}{2}}
\num\newcommand{\f13}{\frac{1}{3}}
\num\newcommand{\f14}{\frac{1}{4}}
\num\newcommand{\f15}{\frac{1}{5}}
\num\newcommand{\f16}{\frac{1}{6}}
\num\newcommand{\f17}{\frac{1}{7}}
\num\newcommand{\f18}{\frac{1}{8}}
\num\newcommand{\f19}{\frac{1}{9}}

\num\newcommand{\f22}{\frac{2}{2}}
\num\newcommand{\f23}{\frac{2}{3}}
\num\newcommand{\f24}{\frac{2}{4}}
\num\newcommand{\f25}{\frac{2}{5}}
\num\newcommand{\f26}{\frac{2}{6}}
\num\newcommand{\f27}{\frac{2}{7}}
\num\newcommand{\f28}{\frac{2}{8}}
\num\newcommand{\f29}{\frac{2}{9}}

\num\newcommand{\f32}{\frac{3}{2}}
\num\newcommand{\f33}{\frac{3}{3}}
\num\newcommand{\f34}{\frac{3}{4}}
\num\newcommand{\f35}{\frac{3}{5}}
\num\newcommand{\f36}{\frac{3}{6}}
\num\newcommand{\f37}{\frac{3}{7}}
\num\newcommand{\f38}{\frac{3}{8}}
\num\newcommand{\f39}{\frac{3}{9}}

\num\newcommand{\f42}{\frac{4}{2}}
\num\newcommand{\f43}{\frac{4}{3}}
\num\newcommand{\f44}{\frac{4}{4}}
\num\newcommand{\f45}{\frac{4}{5}}
\num\newcommand{\f46}{\frac{4}{6}}
\num\newcommand{\f47}{\frac{4}{7}}
\num\newcommand{\f48}{\frac{4}{8}}
\num\newcommand{\f49}{\frac{4}{9}}

\num\newcommand{\f52}{\frac{5}{2}}
\num\newcommand{\f53}{\frac{5}{3}}
\num\newcommand{\f54}{\frac{5}{4}}
\num\newcommand{\f55}{\frac{5}{5}}
\num\newcommand{\f56}{\frac{5}{6}}
\num\newcommand{\f57}{\frac{5}{7}}
\num\newcommand{\f58}{\frac{5}{8}}
\num\newcommand{\f59}{\frac{5}{9}}

\num\newcommand{\f62}{\frac{6}{2}}
\num\newcommand{\f63}{\frac{6}{3}}
\num\newcommand{\f64}{\frac{6}{4}}
\num\newcommand{\f65}{\frac{6}{5}}
\num\newcommand{\f66}{\frac{6}{6}}
\num\newcommand{\f67}{\frac{6}{7}}
\num\newcommand{\f68}{\frac{6}{8}}
\num\newcommand{\f69}{\frac{6}{9}}

\num\newcommand{\f72}{\frac{7}{2}}
\num\newcommand{\f73}{\frac{7}{3}}
\num\newcommand{\f74}{\frac{7}{4}}
\num\newcommand{\f75}{\frac{7}{5}}
\num\newcommand{\f76}{\frac{7}{6}}
\num\newcommand{\f77}{\frac{7}{7}}
\num\newcommand{\f78}{\frac{7}{8}}
\num\newcommand{\f79}{\frac{7}{9}}

\num\newcommand{\f82}{\frac{8}{2}}
\num\newcommand{\f83}{\frac{8}{3}}
\num\newcommand{\f84}{\frac{8}{4}}
\num\newcommand{\f85}{\frac{8}{5}}
\num\newcommand{\f86}{\frac{8}{6}}
\num\newcommand{\f87}{\frac{8}{7}}
\num\newcommand{\f88}{\frac{8}{8}}
\num\newcommand{\f89}{\frac{8}{9}}

\num\newcommand{\f92}{\frac{9}{2}}
\num\newcommand{\f93}{\frac{9}{3}}
\num\newcommand{\f94}{\frac{9}{4}}
\num\newcommand{\f95}{\frac{9}{5}}
\num\newcommand{\f96}{\frac{9}{6}}
\num\newcommand{\f97}{\frac{9}{7}}
\num\newcommand{\f98}{\frac{9}{8}}
\num\newcommand{\f99}{\frac{9}{9}}

% --------------------------------------------------
% Cylindrical and Spherical operators
% --------------------------------------------------
% Cylindrical coordinates {r,\phi,z}
\num\newcommand{\cdiv1}[1]{\frac{1}{r}\frac{\partial}{\partial r}(r#1)}
\num\newcommand{\cdiv2}[1]{\frac{1}{r}\frac{\partial}{\partial\phi}(#1)}
\num\newcommand{\cdiv3}[1]{\frac{\partial}{\partial z}(#1)}

%--------------------------------------------------
% matminux
%--------------------------------------------------
% https://tex.stackexchange.com/questions/75545/negative-sign-and-matrix-alignment
\newcommand*{\mm}{%
  \leavevmode
  \hphantom{0}%
  \llap{%
    \settowidth{\dimen0 }{$0$}%
    \resizebox{1.1\dimen0 }{\height}{$-$}%
  }%
}
 in preamble of a latex document

% --------------------------------------------------
% Use some package dependencies
% --------------------------------------------------
\usepackage{amsmath}  % for \boldsymbol, etc.
\usepackage{amsfonts} % for \mathbb, etc.
\usepackage{scalerel,stackengine} % for \reallywidehat{}
\usepackage{graphicx} % for \includegraphics
\usepackage{caption}  % for captioning
\usepackage{mathtools}% for \ceil and \floor
\usepackage{forest}   % for forest environment

% --------------------------------------------------
% Figures and tables
% --------------------------------------------------
% Could use as-is; better for pattern matching

% Image Macro: \img{filename}{caption}
\newcommand{\img}[2]{
	\begin{figure}[H]
	\centering
	\includegraphics[width=0.6\textwidth]{../images/#1}   % first argument is the file
	\caption{#2}                  % second argument is caption
	\label{fig:#1}                % generate label from first argument
	\end{figure} }

% Double Image Macro: \img{file1}{file2}{caption1}{caption2}
\newcommand{\imgtwo}[4]{
	\begin{figure}
	\centering
	\begin{minipage}{.5\textwidth}
		\centering
		\includegraphics[width=0.9\linewidth]{../images/#1}
		\captionof{figure}{#3}
		\label{fig:#1}
	\end{minipage}%
	\begin{minipage}{.5\textwidth}
		\centering
		\includegraphics[width=0.9\linewidth]{../images/#2}
		\captionof{figure}{#4}
		\label{fig:#2}
	\end{minipage}
	\end{figure}
}

% Table Macro: \tab{filename}{caption}
\newcommand{\tab}[2]{
	\begin{table}[H]
	\centering
	\input{#1} 	% first argument is filename
	\caption{#2} 			% second argument is caption
	\label{tab:#1} 			% generate label from filename
	\end{table}
}

% --------------------------------------------------
% Common sets
% --------------------------------------------------
% Lovingly ripped from Art Owen
\def\reals{\mathbb{R}} % Real number symbol
\def\integers{\mathbb{Z}} % Integer symbol
\def\rationals{\mathbb{Q}} % Rational numbers
\def\naturals{\mathbb{N}} % Natural numbers
\def\complex{\mathbb{C}} % Complex numbers
% With exponent
\def\R#1{\mathbb{R}^{#1}}
\def\Z#1{\mathbb{Z}^{#1}}
\def\Q#1{\mathbb{Q}^{#1}}
\def\N#1{\mathbb{N}^{#1}}
\def\C#1{\mathbb{C}^{#1}}

% --------------------------------------------------
% Vectors and Matrices
% --------------------------------------------------
% Vector symbol macros
\newcommand{\vsym}[1]{\boldsymbol{#1}}
\def\v#1{\vsym{#1}} % \v{x} for vector symbol
% Quick letter vectors
\newcommand{\va}{\boldsymbol{a}}
\newcommand{\vb}{\boldsymbol{b}}
\newcommand{\vc}{\boldsymbol{c}}
\newcommand{\vd}{\boldsymbol{d}}
\newcommand{\ve}{\boldsymbol{e}}
\newcommand{\vf}{\boldsymbol{f}}
\newcommand{\vg}{\boldsymbol{g}}
\newcommand{\vh}{\boldsymbol{h}}
\newcommand{\vi}{\boldsymbol{i}}
\newcommand{\vj}{\boldsymbol{j}}
\newcommand{\vk}{\boldsymbol{k}}
\newcommand{\vl}{\boldsymbol{l}}
\newcommand{\vm}{\boldsymbol{m}}
\newcommand{\vn}{\boldsymbol{n}}
\newcommand{\vo}{\boldsymbol{o}}
\newcommand{\vp}{\boldsymbol{p}}
\newcommand{\vq}{\boldsymbol{q}}
\newcommand{\vr}{\boldsymbol{r}}
\newcommand{\vs}{\boldsymbol{s}}
\newcommand{\vt}{\boldsymbol{t}}
\newcommand{\vu}{\boldsymbol{u}}
\newcommand{\vv}{\boldsymbol{v}}
\newcommand{\vw}{\boldsymbol{w}}
\newcommand{\vx}{\boldsymbol{x}}
\newcommand{\vy}{\boldsymbol{y}}
\newcommand{\vz}{\boldsymbol{z}}

% Tilde shortcut
\newcommand{\tl}[1]{\tilde{#1}}
% Vector symbol + tilde macros
\newcommand{\vta}{\tilde{\boldsymbol{a}}}
\newcommand{\vtb}{\tilde{\boldsymbol{b}}}
\newcommand{\vtc}{\tilde{\boldsymbol{c}}}
\newcommand{\vtd}{\tilde{\boldsymbol{d}}}
\newcommand{\vte}{\tilde{\boldsymbol{e}}}
\newcommand{\vtf}{\tilde{\boldsymbol{f}}}
\newcommand{\vtg}{\tilde{\boldsymbol{g}}}
\newcommand{\vth}{\tilde{\boldsymbol{h}}}
\newcommand{\vti}{\tilde{\boldsymbol{i}}}
\newcommand{\vtj}{\tilde{\boldsymbol{j}}}
\newcommand{\vtk}{\tilde{\boldsymbol{k}}}
\newcommand{\vtl}{\tilde{\boldsymbol{l}}}
\newcommand{\vtm}{\tilde{\boldsymbol{m}}}
\newcommand{\vtn}{\tilde{\boldsymbol{n}}}
\newcommand{\vto}{\tilde{\boldsymbol{o}}}
\newcommand{\vtp}{\tilde{\boldsymbol{p}}}
\newcommand{\vtq}{\tilde{\boldsymbol{q}}}
\newcommand{\vtr}{\tilde{\boldsymbol{r}}}
\newcommand{\vts}{\tilde{\boldsymbol{s}}}
\newcommand{\vtt}{\tilde{\boldsymbol{t}}}
\newcommand{\vtu}{\tilde{\boldsymbol{u}}}
\newcommand{\vtv}{\tilde{\boldsymbol{v}}}
\newcommand{\vtw}{\tilde{\boldsymbol{w}}}
\newcommand{\vtx}{\tilde{\boldsymbol{x}}}
\newcommand{\vty}{\tilde{\boldsymbol{y}}}
\newcommand{\vtz}{\tilde{\boldsymbol{z}}}

% Vector symbol + hat macros
\newcommand{\vha}{\hat{\boldsymbol{a}}}
\newcommand{\vhb}{\hat{\boldsymbol{b}}}
\newcommand{\vhc}{\hat{\boldsymbol{c}}}
\newcommand{\vhd}{\hat{\boldsymbol{d}}}
\newcommand{\vhe}{\hat{\boldsymbol{e}}}
\newcommand{\vhf}{\hat{\boldsymbol{f}}}
\newcommand{\vhg}{\hat{\boldsymbol{g}}}
\newcommand{\vhh}{\hat{\boldsymbol{h}}}
\newcommand{\vhi}{\hat{\boldsymbol{i}}}
\newcommand{\vhj}{\hat{\boldsymbol{j}}}
\newcommand{\vhk}{\hat{\boldsymbol{k}}}
\newcommand{\vhl}{\hat{\boldsymbol{l}}}
\newcommand{\vhm}{\hat{\boldsymbol{m}}}
\newcommand{\vhn}{\hat{\boldsymbol{n}}}
\newcommand{\vho}{\hat{\boldsymbol{o}}}
\newcommand{\vhp}{\hat{\boldsymbol{p}}}
\newcommand{\vhq}{\hat{\boldsymbol{q}}}
\newcommand{\vhr}{\hat{\boldsymbol{r}}}
\newcommand{\vhs}{\hat{\boldsymbol{s}}}
\newcommand{\vht}{\hat{\boldsymbol{t}}}
\newcommand{\vhu}{\hat{\boldsymbol{u}}}
\newcommand{\vhv}{\hat{\boldsymbol{v}}}
\newcommand{\vhw}{\hat{\boldsymbol{w}}}
\newcommand{\vhx}{\hat{\boldsymbol{x}}}
\newcommand{\vhy}{\hat{\boldsymbol{y}}}
\newcommand{\vhz}{\hat{\boldsymbol{z}}}

% Matrix symbol
\newcommand{\msym}[1]{\boldsymbol{#1}}
\def\m#1{\msym{#1}} % short-shortcut

\newcommand{\mA}{\boldsymbol{A}}
\newcommand{\mB}{\boldsymbol{B}}
\newcommand{\mC}{\boldsymbol{C}}
\newcommand{\mD}{\boldsymbol{D}}
\newcommand{\mE}{\boldsymbol{E}}
\newcommand{\mF}{\boldsymbol{F}}
\newcommand{\mG}{\boldsymbol{G}}
\newcommand{\mH}{\boldsymbol{H}}
\newcommand{\mI}{\boldsymbol{I}}
\newcommand{\mJ}{\boldsymbol{J}}
\newcommand{\mK}{\boldsymbol{K}}
\newcommand{\mL}{\boldsymbol{L}}
\newcommand{\mM}{\boldsymbol{M}}
\newcommand{\mN}{\boldsymbol{N}}
\newcommand{\mO}{\boldsymbol{O}}
\newcommand{\mP}{\boldsymbol{P}}
\newcommand{\mQ}{\boldsymbol{Q}}
\newcommand{\mR}{\boldsymbol{R}}
\newcommand{\mS}{\boldsymbol{S}}
\newcommand{\mT}{\boldsymbol{T}}
\newcommand{\mU}{\boldsymbol{U}}
\newcommand{\mV}{\boldsymbol{V}}
\newcommand{\mW}{\boldsymbol{W}}
\newcommand{\mX}{\boldsymbol{X}}
\newcommand{\mY}{\boldsymbol{Y}}
\newcommand{\mZ}{\boldsymbol{Z}}

% Tilde over letter
\newcommand{\tla}{\tilde{a}}
\newcommand{\tlb}{\tilde{b}}
\newcommand{\tlc}{\tilde{c}}
\newcommand{\tld}{\tilde{d}}
\newcommand{\tle}{\tilde{e}}
\newcommand{\tlf}{\tilde{f}}
\newcommand{\tlg}{\tilde{g}}
\newcommand{\tlh}{\tilde{h}}
\newcommand{\tli}{\tilde{i}}
\newcommand{\tlj}{\tilde{j}}
\newcommand{\tlk}{\tilde{k}}
\newcommand{\tll}{\tilde{l}}
\newcommand{\tlm}{\tilde{m}}
\newcommand{\tln}{\tilde{n}}
\newcommand{\tlo}{\tilde{o}}
\newcommand{\tlp}{\tilde{p}}
\newcommand{\tlq}{\tilde{q}}
\newcommand{\tlr}{\tilde{r}}
\newcommand{\tls}{\tilde{s}}
\newcommand{\tlt}{\tilde{t}}
\newcommand{\tlu}{\tilde{u}}
\newcommand{\tlv}{\tilde{v}}
\newcommand{\tlw}{\tilde{w}}
\newcommand{\tlx}{\tilde{x}}
\newcommand{\tly}{\tilde{y}}
\newcommand{\tlz}{\tilde{z}}

% Caligraphic symbol
\def\c#1{\mathcal{#1}} % short-shortcut

\newcommand{\cA}{\mathcal{A}}
\newcommand{\cB}{\mathcal{B}}
\newcommand{\cC}{\mathcal{C}}
\newcommand{\cD}{\mathcal{D}}
\newcommand{\cE}{\mathcal{E}}
\newcommand{\cF}{\mathcal{F}}
\newcommand{\cG}{\mathcal{G}}
\newcommand{\cH}{\mathcal{H}}
\newcommand{\cI}{\mathcal{I}}
\newcommand{\cJ}{\mathcal{J}}
\newcommand{\cK}{\mathcal{K}}
\newcommand{\cL}{\mathcal{L}}
\newcommand{\cM}{\mathcal{M}}
\newcommand{\cN}{\mathcal{N}}
\newcommand{\cO}{\mathcal{O}}
\newcommand{\cP}{\mathcal{P}}
\newcommand{\cQ}{\mathcal{Q}}
\newcommand{\cR}{\mathcal{R}}
\newcommand{\cS}{\mathcal{S}}
\newcommand{\cT}{\mathcal{T}}
\newcommand{\cU}{\mathcal{U}}
\newcommand{\cV}{\mathcal{V}}
\newcommand{\cW}{\mathcal{W}}
\newcommand{\cX}{\mathcal{X}}
\newcommand{\cY}{\mathcal{Y}}
\newcommand{\cZ}{\mathcal{Z}}
% --------------------------------------------------
% Common probability symbols
% --------------------------------------------------
% Lovingly ripped from Art Owen
\newcommand{\mrm}{\mathrm}
\def\E{\mathbb{E}} % Expectation symbol
\def\Earg#1{\E\left[{#1}\right]}
\def\Esubarg#1#2{\E_{#1}\left[{#2}\right]}
\def\P{\mathbb{P}} % Probability symbol
\def\Parg#1{\P\left({#1}\right)}
\def\Psubarg#1#2{\P_{#1}\left[{#2}\right]}
\def\Cov{\mrm{Cov}} % Covariance symbol
\def\Covarg#1{\Cov\left[{#1}\right]}
\def\Covsubarg#1#2{\Cov_{#1}\left[{#2}\right]}
\newcommand{\family}{\mathcal{P}} % probability family / statistical model
\newcommand{\iid}{\stackrel{\mathrm{iid}}{\sim}}
\newcommand{\ind}{\stackrel{\mathrm{ind}}{\sim}}
\def\V{\mathrm{V}} % Variance symbol

% --------------------------------------------------
% Misc
% --------------------------------------------------
% Indicator function
\def\i1{\mathbb{1}}

% Angle bracket average
\newcommand{\avg}[1]{\left\langle#1\right\rangle}

% Markless footnote
% https://tex.stackexchange.com/questions/30720/footnote-without-a-marker
\newcommand\blfootnote[1]{%
  \begingroup
  \renewcommand\thefootnote{}\footnote{#1}%
  \addtocounter{footnote}{-1}%
  \endgroup
}

% Floor and ceiling
% http://tex.stackexchange.com/questions/118173/how-to-write-ceil-and-floor-in-latex
\DeclarePairedDelimiter\ceil{\lceil}{\rceil}
\DeclarePairedDelimiter\floor{\lfloor}{\rfloor}

% Comical & useful reallywidehat
\stackMath
\newcommand\reallywidehat[1]{%
\savestack{\tmpbox}{\stretchto{%
  \scaleto{%
    \scalerel*[\widthof{\ensuremath{#1}}]{\kern-.6pt\bigwedge\kern-.6pt}%
    {\rule[-\textheight/2]{1ex}{\textheight}}%WIDTH-LIMITED BIG WEDGE
  }{\textheight}%
}{0.5ex}}%
\stackon[1pt]{#1}{\tmpbox}%
}

% Comical & useful reallywideparen
\newcommand\reallywideparen[1]{%
\begin{array}{c}
\stretchto{
  \scaleto{
    \scalerel*[\widthof{#1}]{\frown}
    {\rule[-\textheight/2]{1ex}{\textheight}} %WIDTH-LIMITED BIG WEDGE
  }{1.25\textheight} % THIS STRETCHES THE WEDGE A LITTLE EXTRA WIDE
}{0.5ex}\\           % THIS SQUEEZES THE WEDGE TO 0.5ex HEIGHT
#1\\                   % THIS STACKS THE WEDGE ATOP THE ARGUMENT
\rule{0ex}{.01ex}
\end{array}
}
% Useful for debugging; prints to document whether command has been defined already
% Via: http://tex.stackexchange.com/questions/30483/how-can-i-check-in-latex-or-plain-tex-whether-a-command-exists-by-name
\newcommand{\checkfor}[1]{%
  \ifcsname#1\endcsname%
    ... command '#1' exists ...%
  \else%
    ... command '#1' does not exist ...%
  \fi%
}

% Use a forest environment to depict a directory tree
% https://tex.stackexchange.com/questions/5073/making-a-simple-directory-tree
\newcommand{\ftree}[1]{%
\begin{forest}
for tree={
    font=\ttfamily,
    grow'=0,
    child anchor=west,
    parent anchor=south,
    anchor=west,
    calign=first,
    edge path={
      \noexpand\path [draw, \forestoption{edge}]
      (!u.south west) +(7.5pt,0) |- node[fill,inner sep=1.25pt] {} (.child anchor)\forestoption{edge label};
    },
    before typesetting nodes={
      if n=1
        {insert before={[,phantom]}}
        {}
    },
    fit=band,
    before computing xy={l=15pt},
  }%
  #1%
\end{forest}
}

% --------------------------------------------------
% Fractions
% --------------------------------------------------
\usepackage{numdef}   % prefixing with \num allows numbers at end of \newcommand name
\num\newcommand{\f12}{\frac{1}{2}}
\num\newcommand{\f13}{\frac{1}{3}}
\num\newcommand{\f14}{\frac{1}{4}}
\num\newcommand{\f15}{\frac{1}{5}}
\num\newcommand{\f16}{\frac{1}{6}}
\num\newcommand{\f17}{\frac{1}{7}}
\num\newcommand{\f18}{\frac{1}{8}}
\num\newcommand{\f19}{\frac{1}{9}}

\num\newcommand{\f22}{\frac{2}{2}}
\num\newcommand{\f23}{\frac{2}{3}}
\num\newcommand{\f24}{\frac{2}{4}}
\num\newcommand{\f25}{\frac{2}{5}}
\num\newcommand{\f26}{\frac{2}{6}}
\num\newcommand{\f27}{\frac{2}{7}}
\num\newcommand{\f28}{\frac{2}{8}}
\num\newcommand{\f29}{\frac{2}{9}}

\num\newcommand{\f32}{\frac{3}{2}}
\num\newcommand{\f33}{\frac{3}{3}}
\num\newcommand{\f34}{\frac{3}{4}}
\num\newcommand{\f35}{\frac{3}{5}}
\num\newcommand{\f36}{\frac{3}{6}}
\num\newcommand{\f37}{\frac{3}{7}}
\num\newcommand{\f38}{\frac{3}{8}}
\num\newcommand{\f39}{\frac{3}{9}}

\num\newcommand{\f42}{\frac{4}{2}}
\num\newcommand{\f43}{\frac{4}{3}}
\num\newcommand{\f44}{\frac{4}{4}}
\num\newcommand{\f45}{\frac{4}{5}}
\num\newcommand{\f46}{\frac{4}{6}}
\num\newcommand{\f47}{\frac{4}{7}}
\num\newcommand{\f48}{\frac{4}{8}}
\num\newcommand{\f49}{\frac{4}{9}}

\num\newcommand{\f52}{\frac{5}{2}}
\num\newcommand{\f53}{\frac{5}{3}}
\num\newcommand{\f54}{\frac{5}{4}}
\num\newcommand{\f55}{\frac{5}{5}}
\num\newcommand{\f56}{\frac{5}{6}}
\num\newcommand{\f57}{\frac{5}{7}}
\num\newcommand{\f58}{\frac{5}{8}}
\num\newcommand{\f59}{\frac{5}{9}}

\num\newcommand{\f62}{\frac{6}{2}}
\num\newcommand{\f63}{\frac{6}{3}}
\num\newcommand{\f64}{\frac{6}{4}}
\num\newcommand{\f65}{\frac{6}{5}}
\num\newcommand{\f66}{\frac{6}{6}}
\num\newcommand{\f67}{\frac{6}{7}}
\num\newcommand{\f68}{\frac{6}{8}}
\num\newcommand{\f69}{\frac{6}{9}}

\num\newcommand{\f72}{\frac{7}{2}}
\num\newcommand{\f73}{\frac{7}{3}}
\num\newcommand{\f74}{\frac{7}{4}}
\num\newcommand{\f75}{\frac{7}{5}}
\num\newcommand{\f76}{\frac{7}{6}}
\num\newcommand{\f77}{\frac{7}{7}}
\num\newcommand{\f78}{\frac{7}{8}}
\num\newcommand{\f79}{\frac{7}{9}}

\num\newcommand{\f82}{\frac{8}{2}}
\num\newcommand{\f83}{\frac{8}{3}}
\num\newcommand{\f84}{\frac{8}{4}}
\num\newcommand{\f85}{\frac{8}{5}}
\num\newcommand{\f86}{\frac{8}{6}}
\num\newcommand{\f87}{\frac{8}{7}}
\num\newcommand{\f88}{\frac{8}{8}}
\num\newcommand{\f89}{\frac{8}{9}}

\num\newcommand{\f92}{\frac{9}{2}}
\num\newcommand{\f93}{\frac{9}{3}}
\num\newcommand{\f94}{\frac{9}{4}}
\num\newcommand{\f95}{\frac{9}{5}}
\num\newcommand{\f96}{\frac{9}{6}}
\num\newcommand{\f97}{\frac{9}{7}}
\num\newcommand{\f98}{\frac{9}{8}}
\num\newcommand{\f99}{\frac{9}{9}}

% --------------------------------------------------
% Cylindrical and Spherical operators
% --------------------------------------------------
% Cylindrical coordinates {r,\phi,z}
\num\newcommand{\cdiv1}[1]{\frac{1}{r}\frac{\partial}{\partial r}(r#1)}
\num\newcommand{\cdiv2}[1]{\frac{1}{r}\frac{\partial}{\partial\phi}(#1)}
\num\newcommand{\cdiv3}[1]{\frac{\partial}{\partial z}(#1)}

%--------------------------------------------------
% matminux
%--------------------------------------------------
% https://tex.stackexchange.com/questions/75545/negative-sign-and-matrix-alignment
\newcommand*{\mm}{%
  \leavevmode
  \hphantom{0}%
  \llap{%
    \settowidth{\dimen0 }{$0$}%
    \resizebox{1.1\dimen0 }{\height}{$-$}%
  }%
}
 in preamble of a latex document

% --------------------------------------------------
% Use some package dependencies
% --------------------------------------------------
\usepackage{amsmath}  % for \boldsymbol, etc.
\usepackage{amsfonts} % for \mathbb, etc.
\usepackage{scalerel,stackengine} % for \reallywidehat{}
\usepackage{graphicx} % for \includegraphics
\usepackage{caption}  % for captioning
\usepackage{mathtools}% for \ceil and \floor
\usepackage{forest}   % for forest environment

% --------------------------------------------------
% Figures and tables
% --------------------------------------------------
% Could use as-is; better for pattern matching

% Image Macro: \img{filename}{caption}
\newcommand{\img}[2]{
	\begin{figure}[H]
	\centering
	\includegraphics[width=0.6\textwidth]{../images/#1}   % first argument is the file
	\caption{#2}                  % second argument is caption
	\label{fig:#1}                % generate label from first argument
	\end{figure} }

% Double Image Macro: \img{file1}{file2}{caption1}{caption2}
\newcommand{\imgtwo}[4]{
	\begin{figure}
	\centering
	\begin{minipage}{.5\textwidth}
		\centering
		\includegraphics[width=0.9\linewidth]{../images/#1}
		\captionof{figure}{#3}
		\label{fig:#1}
	\end{minipage}%
	\begin{minipage}{.5\textwidth}
		\centering
		\includegraphics[width=0.9\linewidth]{../images/#2}
		\captionof{figure}{#4}
		\label{fig:#2}
	\end{minipage}
	\end{figure}
}

% Table Macro: \tab{filename}{caption}
\newcommand{\tab}[2]{
	\begin{table}[H]
	\centering
	\input{#1} 	% first argument is filename
	\caption{#2} 			% second argument is caption
	\label{tab:#1} 			% generate label from filename
	\end{table}
}

% --------------------------------------------------
% Common sets
% --------------------------------------------------
% Lovingly ripped from Art Owen
\def\reals{\mathbb{R}} % Real number symbol
\def\integers{\mathbb{Z}} % Integer symbol
\def\rationals{\mathbb{Q}} % Rational numbers
\def\naturals{\mathbb{N}} % Natural numbers
\def\complex{\mathbb{C}} % Complex numbers
% With exponent
\def\R#1{\mathbb{R}^{#1}}
\def\Z#1{\mathbb{Z}^{#1}}
\def\Q#1{\mathbb{Q}^{#1}}
\def\N#1{\mathbb{N}^{#1}}
\def\C#1{\mathbb{C}^{#1}}

% --------------------------------------------------
% Vectors and Matrices
% --------------------------------------------------
% Vector symbol macros
\newcommand{\vsym}[1]{\boldsymbol{#1}}
\def\v#1{\vsym{#1}} % \v{x} for vector symbol
% Quick letter vectors
\newcommand{\va}{\boldsymbol{a}}
\newcommand{\vb}{\boldsymbol{b}}
\newcommand{\vc}{\boldsymbol{c}}
\newcommand{\vd}{\boldsymbol{d}}
\newcommand{\ve}{\boldsymbol{e}}
\newcommand{\vf}{\boldsymbol{f}}
\newcommand{\vg}{\boldsymbol{g}}
\newcommand{\vh}{\boldsymbol{h}}
\newcommand{\vi}{\boldsymbol{i}}
\newcommand{\vj}{\boldsymbol{j}}
\newcommand{\vk}{\boldsymbol{k}}
\newcommand{\vl}{\boldsymbol{l}}
\newcommand{\vm}{\boldsymbol{m}}
\newcommand{\vn}{\boldsymbol{n}}
\newcommand{\vo}{\boldsymbol{o}}
\newcommand{\vp}{\boldsymbol{p}}
\newcommand{\vq}{\boldsymbol{q}}
\newcommand{\vr}{\boldsymbol{r}}
\newcommand{\vs}{\boldsymbol{s}}
\newcommand{\vt}{\boldsymbol{t}}
\newcommand{\vu}{\boldsymbol{u}}
\newcommand{\vv}{\boldsymbol{v}}
\newcommand{\vw}{\boldsymbol{w}}
\newcommand{\vx}{\boldsymbol{x}}
\newcommand{\vy}{\boldsymbol{y}}
\newcommand{\vz}{\boldsymbol{z}}

% Tilde shortcut
\newcommand{\tl}[1]{\tilde{#1}}
% Vector symbol + tilde macros
\newcommand{\vta}{\tilde{\boldsymbol{a}}}
\newcommand{\vtb}{\tilde{\boldsymbol{b}}}
\newcommand{\vtc}{\tilde{\boldsymbol{c}}}
\newcommand{\vtd}{\tilde{\boldsymbol{d}}}
\newcommand{\vte}{\tilde{\boldsymbol{e}}}
\newcommand{\vtf}{\tilde{\boldsymbol{f}}}
\newcommand{\vtg}{\tilde{\boldsymbol{g}}}
\newcommand{\vth}{\tilde{\boldsymbol{h}}}
\newcommand{\vti}{\tilde{\boldsymbol{i}}}
\newcommand{\vtj}{\tilde{\boldsymbol{j}}}
\newcommand{\vtk}{\tilde{\boldsymbol{k}}}
\newcommand{\vtl}{\tilde{\boldsymbol{l}}}
\newcommand{\vtm}{\tilde{\boldsymbol{m}}}
\newcommand{\vtn}{\tilde{\boldsymbol{n}}}
\newcommand{\vto}{\tilde{\boldsymbol{o}}}
\newcommand{\vtp}{\tilde{\boldsymbol{p}}}
\newcommand{\vtq}{\tilde{\boldsymbol{q}}}
\newcommand{\vtr}{\tilde{\boldsymbol{r}}}
\newcommand{\vts}{\tilde{\boldsymbol{s}}}
\newcommand{\vtt}{\tilde{\boldsymbol{t}}}
\newcommand{\vtu}{\tilde{\boldsymbol{u}}}
\newcommand{\vtv}{\tilde{\boldsymbol{v}}}
\newcommand{\vtw}{\tilde{\boldsymbol{w}}}
\newcommand{\vtx}{\tilde{\boldsymbol{x}}}
\newcommand{\vty}{\tilde{\boldsymbol{y}}}
\newcommand{\vtz}{\tilde{\boldsymbol{z}}}

% Vector symbol + hat macros
\newcommand{\vha}{\hat{\boldsymbol{a}}}
\newcommand{\vhb}{\hat{\boldsymbol{b}}}
\newcommand{\vhc}{\hat{\boldsymbol{c}}}
\newcommand{\vhd}{\hat{\boldsymbol{d}}}
\newcommand{\vhe}{\hat{\boldsymbol{e}}}
\newcommand{\vhf}{\hat{\boldsymbol{f}}}
\newcommand{\vhg}{\hat{\boldsymbol{g}}}
\newcommand{\vhh}{\hat{\boldsymbol{h}}}
\newcommand{\vhi}{\hat{\boldsymbol{i}}}
\newcommand{\vhj}{\hat{\boldsymbol{j}}}
\newcommand{\vhk}{\hat{\boldsymbol{k}}}
\newcommand{\vhl}{\hat{\boldsymbol{l}}}
\newcommand{\vhm}{\hat{\boldsymbol{m}}}
\newcommand{\vhn}{\hat{\boldsymbol{n}}}
\newcommand{\vho}{\hat{\boldsymbol{o}}}
\newcommand{\vhp}{\hat{\boldsymbol{p}}}
\newcommand{\vhq}{\hat{\boldsymbol{q}}}
\newcommand{\vhr}{\hat{\boldsymbol{r}}}
\newcommand{\vhs}{\hat{\boldsymbol{s}}}
\newcommand{\vht}{\hat{\boldsymbol{t}}}
\newcommand{\vhu}{\hat{\boldsymbol{u}}}
\newcommand{\vhv}{\hat{\boldsymbol{v}}}
\newcommand{\vhw}{\hat{\boldsymbol{w}}}
\newcommand{\vhx}{\hat{\boldsymbol{x}}}
\newcommand{\vhy}{\hat{\boldsymbol{y}}}
\newcommand{\vhz}{\hat{\boldsymbol{z}}}

% Matrix symbol
\newcommand{\msym}[1]{\boldsymbol{#1}}
\def\m#1{\msym{#1}} % short-shortcut

\newcommand{\mA}{\boldsymbol{A}}
\newcommand{\mB}{\boldsymbol{B}}
\newcommand{\mC}{\boldsymbol{C}}
\newcommand{\mD}{\boldsymbol{D}}
\newcommand{\mE}{\boldsymbol{E}}
\newcommand{\mF}{\boldsymbol{F}}
\newcommand{\mG}{\boldsymbol{G}}
\newcommand{\mH}{\boldsymbol{H}}
\newcommand{\mI}{\boldsymbol{I}}
\newcommand{\mJ}{\boldsymbol{J}}
\newcommand{\mK}{\boldsymbol{K}}
\newcommand{\mL}{\boldsymbol{L}}
\newcommand{\mM}{\boldsymbol{M}}
\newcommand{\mN}{\boldsymbol{N}}
\newcommand{\mO}{\boldsymbol{O}}
\newcommand{\mP}{\boldsymbol{P}}
\newcommand{\mQ}{\boldsymbol{Q}}
\newcommand{\mR}{\boldsymbol{R}}
\newcommand{\mS}{\boldsymbol{S}}
\newcommand{\mT}{\boldsymbol{T}}
\newcommand{\mU}{\boldsymbol{U}}
\newcommand{\mV}{\boldsymbol{V}}
\newcommand{\mW}{\boldsymbol{W}}
\newcommand{\mX}{\boldsymbol{X}}
\newcommand{\mY}{\boldsymbol{Y}}
\newcommand{\mZ}{\boldsymbol{Z}}

% Tilde over letter
\newcommand{\tla}{\tilde{a}}
\newcommand{\tlb}{\tilde{b}}
\newcommand{\tlc}{\tilde{c}}
\newcommand{\tld}{\tilde{d}}
\newcommand{\tle}{\tilde{e}}
\newcommand{\tlf}{\tilde{f}}
\newcommand{\tlg}{\tilde{g}}
\newcommand{\tlh}{\tilde{h}}
\newcommand{\tli}{\tilde{i}}
\newcommand{\tlj}{\tilde{j}}
\newcommand{\tlk}{\tilde{k}}
\newcommand{\tll}{\tilde{l}}
\newcommand{\tlm}{\tilde{m}}
\newcommand{\tln}{\tilde{n}}
\newcommand{\tlo}{\tilde{o}}
\newcommand{\tlp}{\tilde{p}}
\newcommand{\tlq}{\tilde{q}}
\newcommand{\tlr}{\tilde{r}}
\newcommand{\tls}{\tilde{s}}
\newcommand{\tlt}{\tilde{t}}
\newcommand{\tlu}{\tilde{u}}
\newcommand{\tlv}{\tilde{v}}
\newcommand{\tlw}{\tilde{w}}
\newcommand{\tlx}{\tilde{x}}
\newcommand{\tly}{\tilde{y}}
\newcommand{\tlz}{\tilde{z}}

% Caligraphic symbol
\def\c#1{\mathcal{#1}} % short-shortcut

\newcommand{\cA}{\mathcal{A}}
\newcommand{\cB}{\mathcal{B}}
\newcommand{\cC}{\mathcal{C}}
\newcommand{\cD}{\mathcal{D}}
\newcommand{\cE}{\mathcal{E}}
\newcommand{\cF}{\mathcal{F}}
\newcommand{\cG}{\mathcal{G}}
\newcommand{\cH}{\mathcal{H}}
\newcommand{\cI}{\mathcal{I}}
\newcommand{\cJ}{\mathcal{J}}
\newcommand{\cK}{\mathcal{K}}
\newcommand{\cL}{\mathcal{L}}
\newcommand{\cM}{\mathcal{M}}
\newcommand{\cN}{\mathcal{N}}
\newcommand{\cO}{\mathcal{O}}
\newcommand{\cP}{\mathcal{P}}
\newcommand{\cQ}{\mathcal{Q}}
\newcommand{\cR}{\mathcal{R}}
\newcommand{\cS}{\mathcal{S}}
\newcommand{\cT}{\mathcal{T}}
\newcommand{\cU}{\mathcal{U}}
\newcommand{\cV}{\mathcal{V}}
\newcommand{\cW}{\mathcal{W}}
\newcommand{\cX}{\mathcal{X}}
\newcommand{\cY}{\mathcal{Y}}
\newcommand{\cZ}{\mathcal{Z}}
% --------------------------------------------------
% Common probability symbols
% --------------------------------------------------
% Lovingly ripped from Art Owen
\newcommand{\mrm}{\mathrm}
\def\E{\mathbb{E}} % Expectation symbol
\def\Earg#1{\E\left[{#1}\right]}
\def\Esubarg#1#2{\E_{#1}\left[{#2}\right]}
\def\P{\mathbb{P}} % Probability symbol
\def\Parg#1{\P\left({#1}\right)}
\def\Psubarg#1#2{\P_{#1}\left[{#2}\right]}
\def\Cov{\mrm{Cov}} % Covariance symbol
\def\Covarg#1{\Cov\left[{#1}\right]}
\def\Covsubarg#1#2{\Cov_{#1}\left[{#2}\right]}
\newcommand{\family}{\mathcal{P}} % probability family / statistical model
\newcommand{\iid}{\stackrel{\mathrm{iid}}{\sim}}
\newcommand{\ind}{\stackrel{\mathrm{ind}}{\sim}}
\def\V{\mathrm{V}} % Variance symbol

% --------------------------------------------------
% Misc
% --------------------------------------------------
% Angle bracket average
\newcommand{\avg}[1]{\left\langle#1\right\rangle}

% Markless footnote
% https://tex.stackexchange.com/questions/30720/footnote-without-a-marker
\newcommand\blfootnote[1]{%
  \begingroup
  \renewcommand\thefootnote{}\footnote{#1}%
  \addtocounter{footnote}{-1}%
  \endgroup
}

% Floor and ceiling
% http://tex.stackexchange.com/questions/118173/how-to-write-ceil-and-floor-in-latex
\DeclarePairedDelimiter\ceil{\lceil}{\rceil}
\DeclarePairedDelimiter\floor{\lfloor}{\rfloor}

% Comical & useful reallywidehat
\stackMath
\newcommand\reallywidehat[1]{%
\savestack{\tmpbox}{\stretchto{%
  \scaleto{%
    \scalerel*[\widthof{\ensuremath{#1}}]{\kern-.6pt\bigwedge\kern-.6pt}%
    {\rule[-\textheight/2]{1ex}{\textheight}}%WIDTH-LIMITED BIG WEDGE
  }{\textheight}%
}{0.5ex}}%
\stackon[1pt]{#1}{\tmpbox}%
}

% Comical & useful reallywideparen
\newcommand\reallywideparen[1]{%
\begin{array}{c}
\stretchto{
  \scaleto{
    \scalerel*[\widthof{#1}]{\frown}
    {\rule[-\textheight/2]{1ex}{\textheight}} %WIDTH-LIMITED BIG WEDGE
  }{1.25\textheight} % THIS STRETCHES THE WEDGE A LITTLE EXTRA WIDE
}{0.5ex}\\           % THIS SQUEEZES THE WEDGE TO 0.5ex HEIGHT
#1\\                   % THIS STACKS THE WEDGE ATOP THE ARGUMENT
\rule{0ex}{.01ex}
\end{array}
}
% Useful for debugging; prints to document whether command has been defined already
% Via: http://tex.stackexchange.com/questions/30483/how-can-i-check-in-latex-or-plain-tex-whether-a-command-exists-by-name
\newcommand{\checkfor}[1]{%
  \ifcsname#1\endcsname%
    ... command '#1' exists ...%
  \else%
    ... command '#1' does not exist ...%
  \fi%
}

% Use a forest environment to depict a directory tree
% https://tex.stackexchange.com/questions/5073/making-a-simple-directory-tree
\newcommand{\ftree}[1]{%
\begin{forest}
for tree={
    font=\ttfamily,
    grow'=0,
    child anchor=west,
    parent anchor=south,
    anchor=west,
    calign=first,
    edge path={
      \noexpand\path [draw, \forestoption{edge}]
      (!u.south west) +(7.5pt,0) |- node[fill,inner sep=1.25pt] {} (.child anchor)\forestoption{edge label};
    },
    before typesetting nodes={
      if n=1
        {insert before={[,phantom]}}
        {}
    },
    fit=band,
    before computing xy={l=15pt},
  }%
  #1%
\end{forest}
}

% --------------------------------------------------
% Fractions
% --------------------------------------------------
\usepackage{numdef}   % prefixing with \num allows numbers at end of \newcommand name
\num\newcommand{\f12}{\frac{1}{2}}
\num\newcommand{\f13}{\frac{1}{3}}
\num\newcommand{\f14}{\frac{1}{4}}
\num\newcommand{\f15}{\frac{1}{5}}
\num\newcommand{\f16}{\frac{1}{6}}
\num\newcommand{\f17}{\frac{1}{7}}
\num\newcommand{\f18}{\frac{1}{8}}
\num\newcommand{\f19}{\frac{1}{9}}

\num\newcommand{\f22}{\frac{2}{2}}
\num\newcommand{\f23}{\frac{2}{3}}
\num\newcommand{\f24}{\frac{2}{4}}
\num\newcommand{\f25}{\frac{2}{5}}
\num\newcommand{\f26}{\frac{2}{6}}
\num\newcommand{\f27}{\frac{2}{7}}
\num\newcommand{\f28}{\frac{2}{8}}
\num\newcommand{\f29}{\frac{2}{9}}

\num\newcommand{\f32}{\frac{3}{2}}
\num\newcommand{\f33}{\frac{3}{3}}
\num\newcommand{\f34}{\frac{3}{4}}
\num\newcommand{\f35}{\frac{3}{5}}
\num\newcommand{\f36}{\frac{3}{6}}
\num\newcommand{\f37}{\frac{3}{7}}
\num\newcommand{\f38}{\frac{3}{8}}
\num\newcommand{\f39}{\frac{3}{9}}

\num\newcommand{\f42}{\frac{4}{2}}
\num\newcommand{\f43}{\frac{4}{3}}
\num\newcommand{\f44}{\frac{4}{4}}
\num\newcommand{\f45}{\frac{4}{5}}
\num\newcommand{\f46}{\frac{4}{6}}
\num\newcommand{\f47}{\frac{4}{7}}
\num\newcommand{\f48}{\frac{4}{8}}
\num\newcommand{\f49}{\frac{4}{9}}

\num\newcommand{\f52}{\frac{5}{2}}
\num\newcommand{\f53}{\frac{5}{3}}
\num\newcommand{\f54}{\frac{5}{4}}
\num\newcommand{\f55}{\frac{5}{5}}
\num\newcommand{\f56}{\frac{5}{6}}
\num\newcommand{\f57}{\frac{5}{7}}
\num\newcommand{\f58}{\frac{5}{8}}
\num\newcommand{\f59}{\frac{5}{9}}

\num\newcommand{\f62}{\frac{6}{2}}
\num\newcommand{\f63}{\frac{6}{3}}
\num\newcommand{\f64}{\frac{6}{4}}
\num\newcommand{\f65}{\frac{6}{5}}
\num\newcommand{\f66}{\frac{6}{6}}
\num\newcommand{\f67}{\frac{6}{7}}
\num\newcommand{\f68}{\frac{6}{8}}
\num\newcommand{\f69}{\frac{6}{9}}

\num\newcommand{\f72}{\frac{7}{2}}
\num\newcommand{\f73}{\frac{7}{3}}
\num\newcommand{\f74}{\frac{7}{4}}
\num\newcommand{\f75}{\frac{7}{5}}
\num\newcommand{\f76}{\frac{7}{6}}
\num\newcommand{\f77}{\frac{7}{7}}
\num\newcommand{\f78}{\frac{7}{8}}
\num\newcommand{\f79}{\frac{7}{9}}

\num\newcommand{\f82}{\frac{8}{2}}
\num\newcommand{\f83}{\frac{8}{3}}
\num\newcommand{\f84}{\frac{8}{4}}
\num\newcommand{\f85}{\frac{8}{5}}
\num\newcommand{\f86}{\frac{8}{6}}
\num\newcommand{\f87}{\frac{8}{7}}
\num\newcommand{\f88}{\frac{8}{8}}
\num\newcommand{\f89}{\frac{8}{9}}

\num\newcommand{\f92}{\frac{9}{2}}
\num\newcommand{\f93}{\frac{9}{3}}
\num\newcommand{\f94}{\frac{9}{4}}
\num\newcommand{\f95}{\frac{9}{5}}
\num\newcommand{\f96}{\frac{9}{6}}
\num\newcommand{\f97}{\frac{9}{7}}
\num\newcommand{\f98}{\frac{9}{8}}
\num\newcommand{\f99}{\frac{9}{9}}

% --------------------------------------------------
% Cylindrical and Spherical operators
% --------------------------------------------------
% Cylindrical coordinates {r,\phi,z}
\num\newcommand{\cdiv1}[1]{\frac{1}{r}\frac{\partial}{\partial r}(r#1)}
\num\newcommand{\cdiv2}[1]{\frac{1}{r}\frac{\partial}{\partial\phi}(#1)}
\num\newcommand{\cdiv3}[1]{\frac{\partial}{\partial z}(#1)}

%--------------------------------------------------
% matminux
%--------------------------------------------------
% https://tex.stackexchange.com/questions/75545/negative-sign-and-matrix-alignment
\newcommand*{\mm}{%
  \leavevmode
  \hphantom{0}%
  \llap{%
    \settowidth{\dimen0 }{$0$}%
    \resizebox{1.1\dimen0 }{\height}{$-$}%
  }%
}
